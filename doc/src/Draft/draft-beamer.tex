
% LaTeX Beamer file automatically generated from DocOnce
% https://github.com/hplgit/doconce

%-------------------- begin beamer-specific preamble ----------------------

\documentclass{beamer}

\usetheme{red_plain}
\usecolortheme{default}

% turn off the almost invisible, yet disturbing, navigation symbols:
\setbeamertemplate{navigation symbols}{}

% Examples on customization:
%\usecolortheme[named=RawSienna]{structure}
%\usetheme[height=7mm]{Rochester}
%\setbeamerfont{frametitle}{family=\rmfamily,shape=\itshape}
%\setbeamertemplate{items}[ball]
%\setbeamertemplate{blocks}[rounded][shadow=true]
%\useoutertheme{infolines}
%
%\usefonttheme{}
%\useinntertheme{}
%
%\setbeameroption{show notes}
%\setbeameroption{show notes on second screen=right}

% fine for B/W printing:
%\usecolortheme{seahorse}

\usepackage{pgf}
\usepackage{graphicx}
\usepackage{epsfig}
\usepackage{relsize}

\usepackage{fancybox}  % make sure fancybox is loaded before fancyvrb

\usepackage{fancyvrb}
%\usepackage{minted} % requires pygments and latex -shell-escape filename
%\usepackage{anslistings}
%\usepackage{listingsutf8}

\usepackage{amsmath,amssymb,bm}
%\usepackage[latin1]{inputenc}
\usepackage[T1]{fontenc}
\usepackage[utf8]{inputenc}
\usepackage{colortbl}
\usepackage[english]{babel}
\usepackage{tikz}
\usepackage{framed}
% Use some nice templates
\beamertemplatetransparentcovereddynamic

% --- begin table of contents based on sections ---
% Delete this, if you do not want the table of contents to pop up at
% the beginning of each section:
% (Only section headings can enter the table of contents in Beamer
% slides generated from DocOnce source, while subsections are used
% for the title in ordinary slides.)
\AtBeginSection[]
{
  \begin{frame}<beamer>[plain]
  \frametitle{}
  %\frametitle{Outline}
  \tableofcontents[currentsection]
  \end{frame}
}
% --- end table of contents based on sections ---

% If you wish to uncover everything in a step-wise fashion, uncomment
% the following command:

%\beamerdefaultoverlayspecification{<+->}

\newcommand{\shortinlinecomment}[3]{\note{\textbf{#1}: #2}}
\newcommand{\longinlinecomment}[3]{\shortinlinecomment{#1}{#2}{#3}}

\definecolor{linkcolor}{rgb}{0,0,0.4}
\hypersetup{
    colorlinks=true,
    linkcolor=linkcolor,
    urlcolor=linkcolor,
    pdfmenubar=true,
    pdftoolbar=true,
    bookmarksdepth=3
    }
\setlength{\parskip}{0pt}  % {1em}

\newenvironment{doconceexercise}{}{}
\newcounter{doconceexercisecounter}
\newenvironment{doconce:movie}{}{}
\newcounter{doconce:movie:counter}

\newcommand{\subex}[1]{\noindent\textbf{#1}}  % for subexercises: a), b), etc

%-------------------- end beamer-specific preamble ----------------------

% Add user's preamble




% insert custom LaTeX commands...

\raggedbottom
\makeindex

%-------------------- end preamble ----------------------

\begin{document}

% matching end for #ifdef PREAMBLE

\newcommand{\exercisesection}[1]{\subsection*{#1}}



% ------------------- main content ----------------------



% ----------------- title -------------------------

\title{Computing Across the Disciplines (CAD); a new Center/Department at the University of Oslo}

% ----------------- author(s) -------------------------

\author{Morten Hjorth-Jensen\inst{1}}
\institute{Department of Physics, University of Oslo, Oslo, Norway  and Michigan State University, USA\inst{1}}
% ----------------- end author(s) -------------------------

\date{A proposal to the board of the Mat-Nat Fakultetet of the University of Oslo; Establish a center by fall 2018 and as a new department by fall 2021
% <optional titlepage figure>
% <optional copyright>
}

\begin{frame}[plain,fragile]
\titlepage
\end{frame}

\begin{frame}[plain,fragile]
\frametitle{Why should we focus on Computational Science and Data Science?}

\begin{itemize}
\item \href{{http://pathways.acm.org/executive-summary.html}}{By 2020, it is expected that one of every two jobs in the STEM fields will be in computing} (Association for Computing Machinery, 2013)

\item Computation is an essential and cross-cutting element of all STEM disciplines

\item Computational science has developed into a discipline of its own right

\item Computations and the understanding of large data sets will play an even larger role in basically all disciplines of STEM fields, Medicine, the Social Sciences, the Humanities and  education

\item Students at both undergraduate and graduate level are unprepared to use computational modeling, data science, and high performance computing – skills valued by a very broad range of employers.
\end{itemize}

\noindent
\end{frame}

\begin{frame}[plain,fragile]
\frametitle{Goals}

\begin{itemize}
\item Position UiO  as a leader in computational science by recruiting faculty whose expertise pertains to large-scale computing and mathematical foundations of data science - both generalists (algorithm/tool developers) and specialists (focused on specific disciplines).	

\item Develop a comprehensive set of courses and degree programs at both undergraduate and graduate levels that will give students across the university exposure to practical computational methods, understanding how to analyse data and more generally to the idea of computers as problem-solving tools.	

\item Facilitate the adoption of computational tools and techniques for both research and education across campus, through education and faculty collaboration. A center and then a department will facilitate the pursuit of these goals!	

\item Educate the next generation of school teachers and university teachers,  with a strong focus on digital competences. 
\end{itemize}

\noindent
\end{frame}

\begin{frame}[plain,fragile]
\frametitle{Strengths, Possibilities and Synergies}

\begin{itemize}
\item Several Centers of excellence in research where Computational Science plays a major role

\item Newly established center of excellence in education research

\item Newly established Master of Science programs in Computational Science and Data Science

\item Several excellent groups in STEM fiels who do Computational Science

\item Computational topics are included in all undergraduate STEM programs, possibility to develop a bachelor program in Computational Science

\item Several educational prizes and awards related to computational science 

\item With a center and later a department we have the possibility to really position UiO as the leading Norwegian and perhaps European institution within Computational Science and Data Science

\item Lead in the development of computations in Life Science

\item Strong links with SIMULA research lab
\end{itemize}

\noindent
\end{frame}

\begin{frame}[plain,fragile]
\frametitle{Enhance Computational Science and Data Science across the disciplines}

Data driven discovery and data driven modeling play already a central role in research. The global objective here is to strengthen and coordinate such activities by bringing together scientists and students across the disciplines.
UiO has already strong computational research and education activities within Mathematics and the Natural Sciences.
The aim here is to extend this to include

\begin{itemize}
\item Computational Science and data science in Mathematics and all of the physical sciences

\item Computational biology and life science (includes medicine)

\item Computational economics and data science and computing in the social science

\item Data science and computing in the Humanities
\end{itemize}

\noindent
The new department will host and coordinate research and educational programs in Computational Science and Data Science. In particular research and education that involve  data analysis and machine Learning will play a central role here. Similarly, this new department will be responsible for developments in quantum computing and quantum information theories.
\end{frame}

\begin{frame}[plain,fragile]
\frametitle{The Center/Department}

\begin{itemize}
\item Administratively located under the Mat-Nat college

\item Composed of 25-30 full time equivalent positions, including some current UiO faculty and when it becomes a department a larger number of new hires.

\item Most of these faculty will have joint appointments with other units and/or departments at the University of Oslo and SIMULA research laboratory. As an example, one can have a 70\% appointment in Mathematics and 30\% at the new department. 

\item Faculty will focus on computational science, data science and large-scale and high-performance computation	

\item Faculty will be incentivized to engage in cross disciplinary and cross-department/college research collaborations	

\item Nurturing environment to attract these faculty and pursue large and interdisciplinary grants	

\item Close ties to SIMULA research laboratory and the HPC center at USIT
\end{itemize}

\noindent
\end{frame}

\begin{frame}[plain,fragile]
\frametitle{Benefits}

\begin{itemize}
\item Recruitment of new faculty who are incentivized to collaborate across the university both in terms of research and education.

\item Opportunities for existing UiO faculty to expand their computation-related capabilities, and to train students to use computational techniques.	

\item Broad and deep educational opportunities for both undergraduate and graduate students across the university.	
\end{itemize}

\noindent
\end{frame}

\begin{frame}[plain,fragile]
\frametitle{Research opportunities}

\begin{itemize}
\item Data driven discovery and data-driven modeling where machine learning plays a central role

\item Research challenges that require computation-oriented multidisciplinary and interdisciplinary approaches.	

\item Research problems that require “bleeding edge” (e.g., multi-petaflop/petabyte) computational approaches to interpret experimental data and complex data.

\item Computational and data science research and education scattered across many departments. New department can strengthen computational science. 	

\item Develop research programs on Quantum Computing, the future of computing.  \href{{https://www.chalmers.se/en/news/Pages/Engineering-of-a-Swedish-quantum-computer-set-to-start.aspx}}{The Wallenberg foundation in Sweden and Chalmers University of technology have funded a project on developing quantum computing technologies with SEK 1 billion}. The aim is to position  Sweden in  a top global top position in quantum technology. 

\item Center-level funding opportunities (e.g., SFF, Marie Curie etc etc).	
\end{itemize}

\noindent
\end{frame}

\begin{frame}[plain,fragile]
\frametitle{More research opportunities}

\begin{itemize}
\item Simulations of complex quantum mechanical systems using novel algorithms, with applications spanning from quantumchromodynamics on the lattice and subatomic physics, via materials to the equation of state of stars.

\item Exploring algorithms from quantum computing in order to solve complicated quantum mechanical problems

\item Study complex materials or the DNA using large-scale molecular dynamics simulations

\item Using machine learning to solve complicated problems, from  neuroscience (our brain), physiology to complicated materials 

\item Using machine learning to develop new tools for learning

\item Bioinformatics, Computational Biology  and Life science

\item Computational economy and computing in the Social Sciences 

\item Data-driven discovery and modeling in  the Humanities
\end{itemize}

\noindent
\end{frame}

\begin{frame}[plain,fragile]
\frametitle{Possible steering committee (2018)}

\begin{enumerate}
\item Institute for theoretical Astrophysics and Rosseland Center for Solar Physics: Mats Carlsson and Viggo Hansteen

\item Bioscience: Tom Andersen and Lex Nederbragt

\item Chemistry and Hylleraas Center for Quantum Molecular Sciences: Michele Cascella, Thomas Bondo Pedersen, Trygve Helgaker and Simen Kvaal

\item Geoscience: John Burkhart, Joe Lacasce and Thomas Vikhamar Schuler

\item IFI Bioinformatics: Torbjørn Rognes

\item IFI Imaging and Biomedical Computing (coupling to Simula): Andreas Austeng, Xing Cai, Joakim Sundnes and Simon Funke

\item Math and Mechanics: Karsten Trulsen and Kent-Andre Mardal, Andreas Carlsson

\item Math and Computational Finance, Statistics and Risk Analysis: Arne Bang Huseby and Geir Olve Storvik

\item Math and Cmputational Mathematics: Geir Dahl, Ragnar Winther, Knut Mørken, Martin Reimers, Michael Floater

\item Physics and Center of Computing in Science Education: Morten Hjorth-Jensen and  Anders Malthe-Sørenssen

\item Medicine: need people

\item UV: Anders Kluge

\item Social Sciences: Kjetil Storsletten and Halvor Mehlum

\item Humanities: need people
\end{enumerate}

\noindent
\end{frame}

\begin{frame}[plain,fragile]
\frametitle{Timeline}

\begin{enumerate}
\item Establish a center called \textbf{Center for Computing across Disciplines} by Fall 2018 and co-locate with the new Center for Computing in Science Education

\item Establish a department  called \textbf{Department of Computational and Data Sciences} by Fall 2021

\item \href{{http://www.uio.no/english/studies/programmes/computational-science-master/index.html}}{New Master of Science Program on Computational Science starts fall 2018}

\item \href{{http://www.uio.no/english/studies/programmes/datascience-master/index.html}}{New Master of Science Program on Data Science starts fall 2018}

\item Extend these Masters progrms to be come cross-college programs

\item Establish  a cross-college PhD program in Computational and Data Sciences, start fall 2020. This PhD program will be a collaboration between the Natural Sciences, Humanities, Social Sciences, Medicine and Education. 

\item Develop a Bachelor program in Computational Science and Data Science? Need to strike a balance between existing programs and possible new \textbf{honors program}. 

\item Submit an application called \textbf{Computing Across the Disciplines} for a Marie Curie training network by spring 2019, 15 PhD positions

\item Submit an application called \textbf{Computing Across the Disciplines} to the Norwegian Research council by Spring 2019, 10 PhD and 10 PD positions

\item Prepare first draft for an SFF application by  spring 2020
\end{enumerate}

\noindent
\end{frame}

\begin{frame}[plain,fragile]
\frametitle{Role and functioning of Center}

For this part, see the long write-up with details and suggestions
\end{frame}

\begin{frame}[plain,fragile]
\frametitle{Role and functioning of Department}

For this part, see the long write-up with details and suggestions
\end{frame}

\begin{frame}[plain,fragile]
\frametitle{Centers and Departments at other universities}

In Norway it is only UiO which offers a Masters program on Computational Science and Data Science. Al other universities have only Master programs on Computer Science. The University of Bergen has a Masters program on Applied Mathematics while UMB has only a Masters on Bioinformatics and Data analysis. These are limited and more focused programs. Nationally, UiO is the only university which offers broad programs in Computational Science and Data Science. 


{\footnotesize
\begin{tabular}{cccccc}
\hline
\multicolumn{1}{c}{ Norway } & \multicolumn{1}{c}{ University } & \multicolumn{1}{c}{ Comp Science and Data dept } & \multicolumn{1}{c}{ Bachelor program } & \multicolumn{1}{c}{ Master program } & \multicolumn{1}{c}{ Graduate/PhD program } \\
\hline
       & UiO        & No                         & No               & Yes                          & No                   \\
       & NTNU       & No                         & No               & No                           & No                   \\
       & UiT        & No                         & No               & No                           & No                   \\
       & UiB        & No                         & No               & No                           & No                   \\
       & UMB        & No                         & No               & Yes (Bioinf+stat datanalyse) & No                   \\
\hline
\end{tabular}
}

\noindent
Out of 95 universities polled in the USA, there are less than 15 which have a department on Scientific Computing
and more than 50 that have a center on Scientific Computing. Between 20 to 30 of these offer a bachelor, Master of Science or PhD program. On Data Science there are approximately 30 departments and 40 centers. Almost 50 of these universities offer a Masters degree in Data Science and close to 40 a PhD in Data Science. 
An excellent example of a department which includes computational science and data science is the newly established \href{{https://cmse.msu.edu/}}{department at Michigan State University}.

At the time of writing, no such poll has been made for European universities. From the list over Masters programs, the countries with the largest focus on these topics are Germany, Sweden and Switzerland. 

The goal in Oslo is to establish a department which covers both Computational Science and Data Science across colleges and disciplines. The department will be responsible for these educational programs and oversee that a coherent and modern selection of courses is offered and developed. The courses should reflect the needs of society at large as well as the specific research projects.  This will give UiO a unique position in Norway.
\end{frame}

\begin{frame}[plain,fragile]
\frametitle{Educational programs at  other universities}

The Society for Industrial and Applied Mathematics (SIAM) \href{{https://www.siam.org/students/resources/cse_programs.php}}{keeps track of graduate programs in computational Science}. The list is most likely not complete.
\end{frame}

\end{document}
